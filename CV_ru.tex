%%%%%%%%%%%%%%%%%
% This is an sample CV template created using altacv.cls
% (v1.6.5, 3 Nov 2022) written by LianTze Lim (liantze@gmail.com). Compiles with pdfLaTeX, XeLaTeX and LuaLaTeX.
%
%% It may be distributed and/or modified under the
%% conditions of the LaTeX Project Public License, either version 1.3
%% of this license or (at your option) any later version.
%% The latest version of this license is in
%%    http://www.latex-project.org/lppl.txt
%% and version 1.3 or later is part of all distributions of LaTeX
%% version 2003/12/01 or later.
%%%%%%%%%%%%%%%%

%% Use the "normalphoto" option if you want a normal photo instead of cropped to a circle
% \documentclass[10pt,a4paper,normalphoto]{altacv}

\documentclass[10pt,a4paper,ragged2e,withhyper]{altacv}
%% AltaCV uses the fontawesome5 and packages.
%% See http://texdoc.net/pkg/fontawesome5 for full list of symbols.

% Change the page layout if you need to
\geometry{left=1.25cm,right=1.25cm,top=1.5cm,bottom=1.5cm,columnsep=1.2cm}

% The paracol package lets you typeset columns of text in parallel
\usepackage{paracol}
\usepackage[english,russian]{babel}
% Change the font if you want to, depending on whether
% you're using pdflatex or xelatex/lualatex
\ifxetexorluatex
  % If using xelatex or lualatex:
  \setmainfont{Roboto Slab}
  \setsansfont{Lato}
  \renewcommand{\familydefault}{\sfdefault}
\else
  % If using pdflatex:
  \usepackage[rm]{roboto}
  \usepackage[defaultsans]{lato}
  % \usepackage{sourcesanspro}
  \renewcommand{\familydefault}{\sfdefault}
\fi

% Change the colours if you want to
\definecolor{SlateGrey}{HTML}{2E2E2E}
\definecolor{LightGrey}{HTML}{666666}
\definecolor{DarkPastelRed}{HTML}{450808}
\definecolor{PastelRed}{HTML}{8F0D0D}
\definecolor{GoldenEarth}{HTML}{E7D192}
\colorlet{name}{black}
\colorlet{tagline}{PastelRed}
\colorlet{heading}{DarkPastelRed}
\colorlet{headingrule}{GoldenEarth}
\colorlet{subheading}{PastelRed}
\colorlet{accent}{PastelRed}
\colorlet{emphasis}{SlateGrey}
\colorlet{body}{LightGrey}

% Change some fonts, if necessary
\renewcommand{\namefont}{\Huge\rmfamily\bfseries}
\renewcommand{\personalinfofont}{\footnotesize}
\renewcommand{\cvsectionfont}{\LARGE\rmfamily\bfseries}
\renewcommand{\cvsubsectionfont}{\large\bfseries}


% Change the bullets for itemize and rating marker
% for \cvskill if you want to
\renewcommand{\itemmarker}{{\small\textbullet}}
\renewcommand{\ratingmarker}{\faCircle}

%% Use (and optionally edit if necessary) this .tex if you
%% want to use an author-year reference style like APA(6)
%% for your publication list
% \input{pubs-authoryear}

%% Use (and optionally edit if necessary) this .tex if you
%% want an originally numerical reference style like IEEE
%% for your publication list
\input{pubs-num}

%% sample.bib contains your publications
\addbibresource{sample.bib}

\begin{document}
\name{Яшин Николай}
\tagline{Ml developer}
%% You can add multiple photos on the left or right
\photoR{2.8cm}{Globe_High}
% \photoL{2.5cm}{Yacht_High,Suitcase_High}

\personalinfo{%
  % Not all of these are required!
  \email{kol.yashin@yandex.ru}
  \phone{8 999 1636057}
  \github{KolyaYashin}
  \printinfo{\faTelegram}{Nikolay}[https://t.me/lelouchviabritannia]
  \printinfo{VK}{Коля Яшин}[https://vk.com/id194353391]
  %% You can add your own arbitrary detail with
  %% \printinfo{symbol}{detail}[optional hyperlink prefix]
  % \printinfo{\faPaw}{Hey ho!}[https://example.com/]
  %% Or you can declare your own field with
  %% \NewInfoFiled{fieldname}{symbol}[optional hyperlink prefix] and use it:
  % \NewInfoField{gitlab}{\faGitlab}[https://gitlab.com/]
  % \gitlab{your_id}
  %%
  %% For services and platforms like Mastodon where there isn't a
  %% straightforward relation between the user ID/nickname and the hyperlink,
  %% you can use \printinfo directly e.g.
  % \printinfo{\faMastodon}{@username@instace}[https://instance.url/@username]
  %% But if you absolutely want to create new dedicated info fields for
  %% such platforms, then use \NewInfoField* with a star:
  % \NewInfoField*{mastodon}{\faMastodon}
  %% then you can use \mastodon, with TWO arguments where the 2nd argument is
  %% the full hyperlink.
  % \mastodon{@username@instance}{https://instance.url/@username}
}

\makecvheader
%% Depending on your tastes, you may want to make fonts of itemize environments slightly smaller
% \AtBeginEnvironment{itemize}{\small}

%% Set the left/right column width ratio to 6:4.
\columnratio{0.6}

% Start a 2-column paracol. Both the left and right columns will automatically
% break across pages if things get too long.
\begin{paracol}{2}
\cvsection{About me}

- Студент 2 курса МГТУ им. Н.Э.Баумана по специальности - "Прикладная математика".\\
- Учусь в ВУЗе на отл, участвую в городских и студенческих олимпиадах по математике.\\
- Прошёл множество курсов посвященным машинному обучению - Deep Learning School на Stepik, курс Воронцова на ютубе, серию курсов от HSE university на Coursera. Ещё понравились хендбуки по ml от Академии Яндекса.\newline
- Также участвовал в хакатонах и соревнованиях, посвященным машинному обучению - например, Цифровой Прорыв 2022, а также соревнования на Kaggle.\\
- В свободное время люблю играть в настольный теннис, в шахматы, а ещё умею собирать кубик рубика с закрытыми глазами.

\cvsection{PET PROJECT}

\cvevent{Vocabulary Bot}{\href{https://github.com/KolyaYashin/VocabBotAiogram/tree/master}{Github}}{}{}
\begin{itemize}
\item Бот, позволяющий пользователям запоминать новые слова, путём добавления их в базу данных 
и закрепления на практике
\item Стэк технологий - Aiogram, Sqlite, TensorFlow
\end{itemize}


\medskip

\cvsection{A Day of My Life}

% Adapted from @Jake's answer from http://tex.stackexchange.com/a/82729/226
% \wheelchart{outer radius}{inner radius}{
% comma-separated list of value/text width/color/detail}
\wheelchart{1.5cm}{0.5cm}{%
  6/8em/accent!30/{8-ми часовой сон},
  3/8em/accent!40/Отдых,
  8/8em/accent!60/Учёба в ВУЗе,
  2/10em/accent/Хобби,
  5/6em/accent!20/Самообразование
}

% use ONLY \newpage if you want to force a page break for
% ONLY the current column
\cvsection{Education}

\cvevent{Бакалавриат}{\href{https://bmstu.ru/}{МГТУ им. Баумана}}{Sept 2021 -- June 2025}{}

\divider

\cvevent{Среднее общее}{\href{https://kpfu.ru/liceum}{Лицей им. Лобачевского при КФУ, Казань}}{Sept 2016 -- June 2021}{}


%% Switch to the right column. This will now automatically move to the second
%% page if the content is too long.
\switchcolumn

\cvsection{My Life Philosophy}

\begin{quote}
``I fear not the man who has practiced 10,000 kicks once, but I fear the man who has practiced one kick 10,000 times.''
\end{quote}

\cvsection{Most Proud of}

\cvachievement{\faTrophy}{\href{https://disk.yandex.ru/d/HK3llnpGPXnfhg}{Дипломы олимпиад}}{Победитель и призёр}

\divider

\cvachievement{\faHeartbeat}{\href{https://lk.hacks-ai.ru/}{Цифровой прорыв 2022}}{Участник}

\divider

\cvachievement{\faHeartbeat}{Золотая медаль в школе}{Для галочки}

\cvsection{Skills}

\cvtag{Python}\\
\cvtag{Numpy}
\cvtag{Pandas}
\cvtag{Aiogram}\\
\cvtag{Sklearn}
\cvtag{TensorFlow}\\
\cvtag{SQLite}
\cvtag{PostrgeSQL}
\cvtag{Docker}\\
\divider
\newline
\cvtag{Time management}
\cvtag{Communication}\\
\cvtag{Flexibility}

\cvsection{Languages}

\cvskill{Русский}{5}
\divider

\cvskill{Английский}{4}
\divider

\cvskill{Татарский}{2}
\divider

\cvskill{Французский}{1.5} %% Supports X.5 values.

%% Yeah I didn't spend too much time making all the
%% spacing consistent... sorry. Use \smallskip, \medskip,
%% \bigskip, \vspace etc to make adjustments.
\medskip


% \divider




\end{paracol}


\end{document}
